\documentclass[12pt]{article} % 12 punto ayarı

% Dil ve Kodlama Ayarları
\usepackage[utf8]{inputenc}
\usepackage[T1]{fontenc}
\usepackage[turkish]{babel} % Türkçe karakter ve heceleme desteği
\usepackage{amsmath}

% Font Ayarları
\usepackage{mathptmx} % Times New Roman fontu için
\usepackage{titlesec}
\usepackage{url}

\usepackage{enumitem} % Listelerin boşluklarını ayarlamak için


% Başlık formatını ayarlar:
% \normalsize = 12 punto (metinle aynı), \bfseries = Kalın yazı
\titleformat{\section}{\normalfont\normalsize\bfseries}{}{0pt}{}

% Başlıkların etrafındaki boşlukları ayarlar:
% {Soldan girinti}{Üst boşluk}{Alt boşluk}
\titlespacing*{\section}{0pt}{12pt}{6pt}



% Sayfa Yapısı Ayarları
% Her kenardan 2.5 cm boşluk
\usepackage[a4paper, left=2.5cm, right=2.5cm, top=2.5cm, bottom=2.5cm]{geometry}

% Diğer Gerekli Paketler
\usepackage{graphicx}
\usepackage{xcolor} % Renkli metinler için
\usepackage{array}  % Tablo düzeni için
\usepackage{longtable} % Sayfa taşan tablolar için (gerekirse)
\usepackage{float}

% Satır Aralığı
\linespread{1.0} % Tek satır aralığı



\usepackage{listings}
\usepackage{xcolor}

% Arduino Kodu İçin Renk Ayarları
\definecolor{codegreen}{rgb}{0,0.6,0}
\definecolor{codegray}{rgb}{0.5,0.5,0.5}
\definecolor{codepurple}{rgb}{0.58,0,0.82}
\definecolor{backcolour}{rgb}{0.95,0.95,0.92}

\lstdefinestyle{mystyle}{
    backgroundcolor=\color{backcolour},   
    commentstyle=\color{codegreen},
    keywordstyle=\color{magenta},
    numberstyle=\tiny\color{codegray},
    stringstyle=\color{codepurple},
    basicstyle=\ttfamily\footnotesize,
    breakatwhitespace=false,         
    breaklines=true,                 
    captionpos=b,                    
    keepspaces=true,                 
    numbers=left,                    
    numbersep=5pt,                  
    showspaces=false,                
    showstringspaces=false,
    showtabs=false,                  
    tabsize=2
}

\lstset{style=mystyle}


\let\oldbibliography\thebibliography
\renewcommand{\thebibliography}[1]{%
  \oldbibliography{#1}%
  \setlength{\itemsep}{0pt}%    % Maddeler arası boşluğu sıfırlar
  \setlength{\parskip}{0pt}%    % Paragraf boşluğunu sıfırlar
}





\begin{document}

\noindent \textbf{Proje Ana Alanı :} Fizik \\[1em]
\textbf{Proje Tematik Alanı :} \\[1em]
\textbf{Proje Adı (Başlığı) : } Nötrino Bozunumunun SU2 uzayında Cart Curt Yapılması \\[2em]

\section*{Özet}{12}
Özet. Burası özet
\textbf{Anahtar Kelimeler:} Kelime 1, Kelime 2

\section*{Amaç}
Burası Amaç. Evet burası gerçekten de amaç

\section*{Giriş}
Burası giriş. Değadasdasdqwesdasdsil mi Ali?

\section*{Yöntem}

\begin{equation}
\frac{1+\sqrt{5}}{2}
\end{equation}


\begin{equation}
   h=6.62607004\times 10^{-34}\tag{21}
\end{equation}


\section*{Proje İş-Zaman Çizelgesi}
% (Mevcut çizelge kodu korundu)
\begin{center}
\renewcommand{\arraystretch}{1.5} % Tablo satır yüksekliği
\begin{tabular}{|p{5cm}|c|c|c|c|c|c|c|c|c|c|}
\hline
\textbf{İşin Adı} & \multicolumn{10}{c|}{\textbf{AYLAR}} \\
\hline
 & \rotatebox{90}{NİSAN} & \rotatebox{90}{MAYIS} & \rotatebox{90}{HAZİRAN} & \rotatebox{90}{TEMMUZ} & \rotatebox{90}{AĞUSTOS} & \rotatebox{90}{EYLÜL} & \rotatebox{90}{EKİM} & \rotatebox{90}{KASIM} & \rotatebox{90}{ARALIK} & \rotatebox{90}{OCAK} \\
\hline
Literatür Taraması & X & X & & & & & & & & \\
\hline
Arazi/Deney Çalışması & & & X & X & X & & & & & \\
\hline
Verilerin Analizi & & & & & & X & X & & & \\
\hline
Proje Raporu Yazımı & & & & & & & & X & X & X \\
\hline
\end{tabular}
\end{center}

\section*{Bulgular}
Bunları şunları bulduk, değil mi Ali? Evet Recep gerçekten bunları şunları bulduk.

\begin{table}[H]
\centering
\begin{tabular}{|c|p{4cm}|c|p{5cm}|}
\hline
\textbf{Deney no} & \textbf{Cart / Curt } & \textbf{Sensör} & \textbf{Sistem Tepkisi} \\
\hline
1 & Su (Referans) & Sensör 3 & Normal Akış (Cart şöyle, Curt şöyle) \\
\hline
2 & Biraz Daha Su & Sensör 3 & Normal Akış \\
\hline
3 & Daha Da Su & Sensör 2 & Tolerans Aralığı (Uyarı Yok) \\
\hline
4 & Çok Daha Su & Sensör 4 & Tolerans Aralığı (Uyarı Yok) \\
\hline
5 & Çok Çok Daha Su & \textbf{Sensör 1} & \textbf{ALARM} (Evet gerçekten de alarm) \\
\hline
6 & Bulanık / Çamurlu Su & \textbf{Sensör 5} & \textbf{ALARM} (Physicist be like Ion now oil industry just pays huge amount. Why not disgrace all my work and kill my future to drive nicer car) \\
\hline
\end{tabular}
\vspace{-4pt}
\caption{Did you know that this is a table}
\end{table}

Deney sonuçları, bubububububububububububububububububububububububububububububububububububu göre avantajını doğrulamaktadır.

\section*{Sonuç ve Tartışma}



\section*{Öneriler}





% Kaynaklar Başlığını Ayarlıyoruz
\renewcommand{\refname}{Kaynaklar} 

% Stil Belirleme: 'plain' alfabetik sıralar ve numaralandırır [1], [2]
\bibliographystyle{plain} 

% .bib dosyasını çağırıyoruz (uzantısız olarak dosya adı)
\bibliography{kaynaklar} 

% DİKKAT: Metin içinde atıf yapmadığın kaynaklar listede çıkmaz.
% Hepsini görmek istiyorsan geçici olarak şu komutu aç:
\nocite{*}




\section*{Ekler}
\textbf{Ek 1: }
Aşağıda, babbubbabubbabbubbabubbabbubbabubbabbubbabub yöneten algoritma verilmiştir.

\begin{lstlisting}[language=C++, caption=Bu bir caption ve cpp dünyanın en iyi dili]
#include <Servo.h>

\end{lstlisting}

\end{document}